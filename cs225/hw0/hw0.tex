\documentclass[11pt]{article}
\usepackage{jeffe,handout}
%\def\rmdefault{bch} % Use Charter for main text font.

\def\BOX#1{\fbox{\vbox to #1{\vss\hbox to #1{\hss}}}}
\def\Bigbox{\BOX{0.25in}}
\def\Bigbox{\raisebox{-0.5ex}[0.25in][0pt]{\BOX{0.25in}}}

\hidesolutions

% =========================================================
\begin{document}

\headers{CS 225}{ }{Spring 2015}

\begin{center}
\LARGE
% \LARGE \textbf{CS 225: Data Structures and Software Principles, Spring
% 2011}
% \\
\textbf{Homework 0}
\\[1ex]
%\Large Due \emph{in lecture} on January 18, 2013.
\Large Due Jan 26, 2015 \\ in lecture and SVN \\ \large Instructions for submission into your \\ class SVN repository are on the webpage.
\end{center}

\bigskip\hrule
\begin{quote}
The purpose of this assignment is to give you  a chance to refresh the math
skills we expect you to have learned in prior classes.  These particular skills
will be essential to  mastery of CS225, and we are unlikely to take much class
time reminding you how to solve similar problems.  Though you are not required
to work independently on this assignment, we encourage you to do so because we
think it may help you diagnose and  remedy some things you might otherwise find
difficult later on in the course. If this homework is difficult, please consider
completing the discrete math requirement (CS173 or MATH 213) before taking CS225. 

%Please identify your paper VERY clearly.  That is, spell your name and netid in capital letters, and tell us the day and time of your section meeting, not just the section name.  Please place this identifying information on the upper right corner of the first page of your work.  Everybody loves a happy grader.
\bigskip
\hrule %\bigskip

\end{quote}
%\begin{description}

\begin{table}[h]
\centering
\renewcommand{\arraystretch}{1.5}
\begin{tabular}{l l }
\bfseries{Name:} & \\
\bfseries{netID:} & \\
\bfseries{Section} (circle one): 
 & \hspace{.2in} AYB Wed 7-9pm, AYC Thurs 9-11am, AYD Thurs 11-1pm, \\
 & \hspace{.2in} AYE Thurs 1-3pm, AYF Thurs 3-5pm, AYG Thurs 5-7pm, \\
 & \hspace{.2in} AYH Thurs 7-9pm, AYI Fri 9-11am, AYJ Fri 11-1pm, \\
 & \hspace{.2in} AYK Fri 1-3pm, AYL Fri 3-5pm, AYM Fri 5-7pm\\
 & Laptop Sections: \\
 & \hspace{.2in} AYN Thur 1-3pm, AYO Thur 5-7pm, AYP Thur 3-5pm, \\
 & \hspace{.2in} AYQ Fri 1-3pm, AYR Fri 9-11am, AYS 11-1pm.\\
\end{tabular}
\end{table}

\vspace{.1in}
\begin{table}[h]
\centering
\renewcommand{\arraystretch}{2}
\begin{tabular}{|l| l| c|}
\hline
Grade & \hspace{.75in} & Out of 60\\
\hline
Grader & \multicolumn{2}{|c|}{}\\
\hline
\end{tabular}
\end{table}

%\end{description}


%\hrule\bigskip
%\vspace{-1in}

\newpage
\begin{problems}
%----------------------------------------------------------------------
\item (3 points) Using 140 characters or less, post a synopsis of your favorite movie to the course piazza space under the ``HW0 tell me something!'' notice, so that your post is visible to everyone in the class, and tagged by \#HW0num1.  Also, use Piazza's code-formatting tools to write a {\em private} post to course staff that includes at least 5 lines of code.  It can be code of your own or from a favorite project--it doesn't even have to be syntactically correct--but it must be formatted as a code block in your post, and also include the tag \#HW0num1.  (Hint: Check https://piazza.com/product/features).  Finally, please write the 2 post numbers corresponding to your posts here:

\begin{table}[h]
\hspace{.3in}  
\renewcommand{\arraystretch}{2}
\begin{tabular}{|l|c| }
\hline
Favorite Movie Post (Public) number: & \hspace{2in} \\
\hline
Formatted Code Post (Private)  number: & \hspace{2in}\\ 
\hline
\end{tabular}
\end{table}

\item (12 points)
Simplify the following expressions as much as possible,
\textbf{without using an calculator (either hardware or software)}.
Do not approximate. Express all rational numbers as improper
fractions. Show your work in the space provided, and write your answer in the box provided.
\begin{enumerate}

\item $\displaystyle\prod_{k=2}^n (1-\dfrac{1}{k^2})$ 
\vspace{1.5in} \begin{table}[!h] \flushright \renewcommand{\arraystretch}{2} \begin{tabular}{|l|c| }
\hline
Answer for (a): & \hspace{2in} \\
\hline
\end{tabular} \end{table}

\item $3^{1000} \bmod 7$ 
\vspace{1.5in}
\begin{table}[!h] \flushright \renewcommand{\arraystretch}{2}  \begin{tabular}{|l|c| } \hline
Answer for (b): & \hspace{2in} \\
\hline \end{tabular} \end{table}

\item $\displaystyle\sum_{r=1}^\infty(\frac{1}{2})^r$
\vspace{1.25in}
\begin{table}[!h] \flushright \renewcommand{\arraystretch}{2}  \begin{tabular}{|l|c| } \hline
Answer for (c): & \hspace{2in} \\
\hline \end{tabular} \end{table}

\item $\dfrac{\log_7 81}{\log_7 9}$ 
\vspace{1in}
\begin{table}[!h] \flushright \renewcommand{\arraystretch}{2}  \begin{tabular}{|l|c| } \hline
Answer for (d): & \hspace{2in} \\
\hline \end{tabular} \end{table}

\item $\log_2 4^{2n}$  
\vspace{1in}
\begin{table}[!h] \flushright \renewcommand{\arraystretch}{2}  \begin{tabular}{|l|c| } \hline
Answer for (e): & \hspace{2in} \\
\hline \end{tabular} \end{table}

\item $\log_{17} 221 - \log_{17} 13$ 
\vspace{1in}
\begin{table}[!h] \flushright \renewcommand{\arraystretch}{2}  \begin{tabular}{|l|c| } \hline
Answer for (f): & \hspace{2in} \\
\hline \end{tabular} \end{table}
\end{enumerate}

\newpage
%----------------------------------------------------------------------
 \item (8 points) Find the formula for $1+$ $\displaystyle\sum_{j=1}^n j!j $,
and show work proving the formula is correct using induction.
\begin{table}[!h] \flushright \renewcommand{\arraystretch}{2}  \begin{tabular}{|l|c| } \hline
Formula: & \hspace{2in} \\
\hline \end{tabular} \end{table}
\vspace{1.75in}

%----------------------------------------------------------------------
% \item If $F_n$ denotes the $n$th term of the Fibonacci sequence,prove that
% $\displaystyle\sum_{n=1}^m F_n = F_{m+2} - 1 $.

%four----------------------------------------------------------------------
\item (8 points) Indicate for each of the following pairs of expressions $(f(n),g(n))$,
whether $f(n)$ is  $O, \Omega,$ or $\Theta$ of $g(n)$.  Prove your answers to the first two items, but just GIVE an answer to the last two.
\begin{enumerate}
\item $f(n)=4^{\log_{4} n}$ and $g(n)=2n+1$.
\vspace{1in}
\begin{table}[!h]  \flushright \renewcommand{\arraystretch}{2}  \begin{tabular}{|l|c| } \hline
Answer for (a): &  \hspace{.15in} $f(n)$ \hspace{.5in} $g(n)$ \hspace{.15in}\\
\hline \end{tabular} \end{table}

\item $f(n)= n^2$ and $g(n)=(\sqrt{2})^{\log_2 n}$.
\vspace{1in}
\begin{table}[!h]  \flushright \renewcommand{\arraystretch}{2}  \begin{tabular}{|l|c| } \hline
Answer for (b): & \hspace{.15in} $f(n)$ \hspace{.5in} $g(n)$ \hspace{.15in}\\
\hline \end{tabular} \end{table}

\item $f(n) =\log_2 n!$ and $g(n) =n \log_2 n$.
\vspace{.25in}
\begin{table}[!h]  \flushright \renewcommand{\arraystretch}{2}  \begin{tabular}{|l|c| } \hline
Answer for (c): & \hspace{.15in} $f(n)$ \hspace{.5in} $g(n)$ \hspace{.15in}\\
\hline \end{tabular} \end{table}

\item $f(n) = n^k$ and $g( n)=c^n$ where $k,c$ are constants and $c$ is $>$1.
\vspace{.25in}
\begin{table}[!h]  \flushright \renewcommand{\arraystretch}{2}  \begin{tabular}{|l|c| } \hline
Answer for (d): &  \hspace{.15in} $f(n)$ \hspace{.5in} $g(n)$ \hspace{.15in}\\
\hline \end{tabular} \end{table}

\end{enumerate}
%five----------------------------------------------------------------------
 \item (9 points) Solve the following recurrence relations for integer $n$. If
no solution exists, please explain the result.
\begin{enumerate}
\item $T(n)=T(\frac{n}{2}) + 5$, $T(1) = 1$, assume $n$ is a power of 2.
\vspace{1.5in}
\begin{table}[!h]  \flushright \renewcommand{\arraystretch}{2}  \begin{tabular}{|l|c| } \hline
Answer for (a): & \hspace{2in} \\
\hline \end{tabular} \end{table}

\item $T(n)= T(n-1) + \frac{1}{n}$, $T(0) = 0$.
\vspace{1.5in}
\begin{table}[!h]  \flushright \renewcommand{\arraystretch}{2}  \begin{tabular}{|l|c| } \hline
Answer for (b): & \hspace{2in} \\
\hline \end{tabular} \end{table}

\item Prove that your answer to part (a) is correct using induction.
\vspace{1.5in}

\end{enumerate}
%----------------------------------------------------------------------

\item (10 points) Suppose function call parameter passing costs constant time, independent of the size of the structure being passed.

\begin{enumerate}
\item Give a recurrence for worst case 
running time of the recursive Binary Search function in terms of $n$, the size of the search array. Assume $n$ is a power of 2. Solve the recurrence.
\vspace{1.25in}
\begin{table}[!h] \flushright \renewcommand{\arraystretch}{2}  \begin{tabular}{|l|c| } \hline
Recurrence: & \hspace{2in} \\
\hline
Base case: & \\
\hline
Recurrence Solution: &  \\
\hline \end{tabular} \end{table}
\item Give a recurrence for worst case 
running time of the recursive Merge Sort function in terms of $n$, the size of the array being sorted. Solve the recurrence. 
\vspace{1.25in}
\begin{table}[!h] \flushright \renewcommand{\arraystretch}{2}  \begin{tabular}{|l|c| } \hline
Recurrence: & \hspace{2in} \\
\hline
Base case: & \\
\hline
Running Time: &  \\
\hline \end{tabular} \end{table}
\end{enumerate}


\newpage
\item (10 points) Consider the pseudocode function below.
\begin{verbatim}
derp( x, n )
     if( n == 0 )
         return 1;
     if( n % 2 == 0 )
         return derp( x^2, n/2 );
     return x * derp( x^2, (n - 1) / 2);
\end{verbatim}
\begin{enumerate}
\item What is the output when passed the following parameters: $x=2$, 
$n=12$. Show your work (activation diagram or similar).

\vspace{.5in}
\begin{table}[!h]  \flushright \renewcommand{\arraystretch}{2}  \begin{tabular}{|l|c| } \hline
Answer for (a): & \hspace{2in} \\
\hline \end{tabular} \end{table}
\item Briefly describe what this function is doing.
\vspace{1in}
\item Write a recurrence that models the running time of this function. 
Assume checks, returns, and arithmetic are constant time, but be sure to 
evaluate all function calls. [Hint: what is the \emph{most} $n$ could be 
at each level of the recurrence?]
\vspace{1in}
\item Solve the above recurrence for the running time of this function.
\vspace{1in}
\end{enumerate}

%----------------------------------------------------------------------
% \item Suppose you and your best friend cannot agree on which movie to see.  To solve the problem you play a best-of-five series of checkers and the winner decides. (i.e. the first player to win 3 games chooses which movie to see.)
%
%\begin{enumerate}
%\item If you win every game with a probability of $\frac{2}{3}$, what is
%the probability that you win the series?
%\item What is the expected number of games in the series if your probability 
%of winning is $\frac{1}{2}$? What if it is $\frac{1}{4}$?

%\end{enumerate}


\end{problems}

\newpage

Blank sheet for extra work.


\end{document}
